\documentclass[sigconf]{acmart}

% Old ACM template no longer used since 2017
%\documentclass{sig-alternate}

\usepackage{multirow}
\usepackage{graphics}
\usepackage{graphicx}
\usepackage{algorithm}
\usepackage{algorithmic}
%\usepackage{algorithm2e}
\usepackage{amsfonts}
\usepackage{amsmath}
\usepackage{color, url}
%\usepackage[justification=centering]{caption}
\usepackage[skip=0pt]{caption}
\usepackage[skip=0pt]{subcaption}
\usepackage{tabto}
\usepackage{url}
\usepackage{xcolor, colortbl}
\usepackage{booktabs}
\usepackage{enumitem}
%\usepackage{subfigure}
\usepackage{booktabs}
\usepackage{amssymb}
\usepackage{url}
\usepackage{balance}



\graphicspath{ {figs/} }


%\newtheorem{theorem}{Theorem}
%\newtheorem{lemma}[theorem]{Lemma}
%\newtheorem{definition}[theorem]{Definition}
%\newtheorem{corollary}[theorem]{Corollary}

%\newtheorem{fact}{Fact}
\DeclareMathOperator*{\argmin}{arg\,min}
\DeclareMathOperator*{\argmax}{arg\,max}

\newcommand{\myparatight}[1]{\smallskip\noindent{\bf {#1}:}~}
%\newcommand{\comment}[1]{{\small\color{red}[COMMENT: #1]}}


\renewcommand{\algorithmicrequire}{\textbf{Input:}}
\renewcommand{\algorithmicensure}{\textbf{Output:}}
\newcommand{\todo}[1]{{\textcolor{red}{{\bf TODO:} #1}}}
\newcommand{\tabincell}[2]{\begin{tabular}{@{}#1@{}}#2\end{tabular}}

\newcommand{\DB}{\ensuremath{\mathcal{ST}}}
\newcommand{\RR}{\ensuremath{\mathbb{R}}}
\newcommand{\NN}{\ensuremath{\mathbb{N}}}
\newcommand{\clust}{\ensuremath{\#\mbox{Clust}}}

\newcommand{\TODO}[1]{{\bf \textcolor[rgb]{1.00,0.00,0.00}{TODO: #1}}}


% ACM info -----------------------------------------------------------------

\iffalse
% Copyright
%\setcopyright{none}
%\setcopyright{acmcopyright}
%\setcopyright{acmlicensed}
\setcopyright{rightsretained}
%\setcopyright{usgov}
%\setcopyright{usgovmixed}
%\setcopyright{cagov}
%\setcopyright{cagovmixed}

% DOI
\acmDOI{10.475/123_4}

% ISBN
\acmISBN{123-4567-24-567/08/09}


%Conference
\acmConference[KDD'17]{ACM SIGKDD Conference on Knowledge Discovery and Data Mining}{August 2017}{Halifax, Nova Scotia Canada}
\acmYear{2017}
\copyrightyear{2017}

\acmPrice{15.00}
\fi

% -----------------------------------------------------------------


\begin{document}


%save space
%\setlength{\abovedisplayskip}{1.5mm}
%\setlength{\belowdisplayskip}{1.5mm}



\title{Spatio-Temporal Prediction of Social Connections}


\author{Guolei Yang}
\affiliation{%
  \institution{Iowa State University}
  \city{Ames} 
  \state{Iowa, USA} 
  \postcode{50011}
}
\email{yanggl@iastate.edu}

\author{Andreas Z\"ufle}
\affiliation{%
  \institution{George Mason University}
  \city{Fairfax} 
  \state{Virginia, USA} 
  \postcode{22030}
}
\email{azufle@gmu.edu}


\begin{abstract}
It is long known that a user's mobility pattern can be affected by his social connections. Users tend to visit same locations visited by their friends. In this paper we investigate the inverse problem: How does a set of users' trajectory reflects their social connections. To this end, we define the social connection prediction problem. Given two users, predict the probability that they are friends by mining their historical trajectories. A naive method to do so is to exam how often the two users visits the same location at the same time, which suffers from the problem that different locations/times may have different predictive power. We propose a comprehensive prediction model that is able to capture this difference between locations and time slots. To demonstrate its effectiveness, we trained the proposed model using the publicly available Foursquare dataset. The result shows the proposed model is able to predict existence of social connections between randomly selected users significantly more accurate comparing with the naive method.
\end{abstract}


\begin{CCSXML}
<ccs2012>
<concept>
<concept_id>10002951.10003227.10003236.10003101</concept_id>
<concept_desc>Information systems~Location based services</concept_desc>
<concept_significance>500</concept_significance>
</concept>
<concept>
<concept_id>10002951.10003227.10003351</concept_id>
<concept_desc>Information systems~Data mining</concept_desc>
<concept_significance>500</concept_significance>
</concept>
<concept>
<concept_id>10010147.10010257</concept_id>
<concept_desc>Computing methodologies~Machine learning</concept_desc>
<concept_significance>300</concept_significance>
</concept>
</ccs2012>
\end{CCSXML}

\ccsdesc[500]{Information systems~Location based services}
\ccsdesc[500]{Information systems~Data mining}
\ccsdesc[300]{Computing methodologies~Machine learning}


\keywords{Location-Based Social Network, social connection prediction, feature selection, spatio-temporal data}


\maketitle


\section{Introduction}\label{sec:intro}

In the past decade, with the rise of Location-Based Social Networks (LBSN), huge amount of geo-spatial data is collected on a daily basis. For example, the Foursquare\cite{yang2015nationtelescope} dataset contains more than 30 millions of self-reported check-ins from thousands of user around the world. As a result, it becomes possible to mine spatio-temporal data and study human mobility pattern at unprecedented large scale. 

It is long known that a user's mobility pattern can be affected by his social connections~\cite{cho2011friendship, ye2013s}. For example, a group of close friends tend to check-in to the same locations at the same time period. As such, it is possible to predict a user's future movement by mining the historical trajectory of his friends on LBSN. In the past decade, making predictions with spatio-temporal data has been intensively studied. Existing research mainly focus on the prediction of future movements (e.g., \cite{cho2011friendship, noulas2012mining, gao2012mobile, scellato2011nextplace, lian2013collaborative}). To our knowledge, however, predicting a user's social connections with spatio-temporal data has not been studied in literate. 

Towards the goal of a more thorough understanding of human mobility patterns, we propose to investigate the predictive power of spatio-temporal data in predicting a user's social connections. In particular, given the trajectories of two LBSN users $u_i$ and $u_j$, we aim to predict the probability that $u_i$ and $u_j$ are friends on the LBSN. Social connection prediction is a long standing research topic, mainly used as a tool for friend recommendation on social networks. Most existing methods exploit a user's profile and existing social connections to make friend recommendations, but not the user's trajectory. Our research is not competitive, but supplementary to existing friend recommendation methods.

A straightforward way to predict the social connection, or the lack thereof, between two users is to exam the \textit{spatio-temporal overlap} of their trajectories, i.e., find events where the two users visit the same location at the same time on their trajectories. Such an event is called a \textit{co-visitation} of the two users. The assumption is, if two users frequently visits the same location during the same time peroid, they might be friend with each other. Thus the number of co-visitations could reflect when and where they were meeting. Algorithms such as co-location mining~\cite{weiler2015geo} can be used to discover such co-visitations among users.

Although the above assumption is reasonable, this naive solution suffers from two problems. First, it treats all locations equally in predicting social connections, which is not realistic. For example, if two users frequently meet at private locations like someone's house, or a small coffee shop, it is very likely that they know each other. However, if they both check-in to the same Walmart supermarket after work, it might be just an coincidence simply because there it is the only supermarket near their home. Second, this method ignores the time difference of check-ins behaviours. If two users both check-in to a restaurant at 6:00pm, it is not as significant as two users visit the same location at 10:00pm. This is because most customer of the restaurant may choose to dine there around 6:00pm, but if two users both decide to check-in there at 10:00pm, the chance that it is purely an coincidence is relatively lower.

We propose to employ a more comprehensive methodology to study the social connection prediction problem. Unlike the naive solution, we assume different locations and different time slots have different predictive power. We propose a social connection prediction model that is able to capture the inherent difference among locations and times. Specifically, each location and time slot is assigned a weight which measures the significance of the location and time slot in predicting social connections. We then use the Foursquare dataset to learn the weights in the proposed model and make predictions using users' social connections on Foursquare as ground truth. We show that the proposed model significantly outperforms the naive algorithm that exams only the number of co-visitations. We summarize our contributions as follows:
\begin{itemize}

\item We study the predictive power of spatio-temporal data in predicting social connections. Given the trajectories of two LBSN users $u$ and $v$, we aim to predict the probability that $u$ and $v$ are friends on the LBSN.

\item We assume different locations and time may have different predictive power, which is in accordance with common sense. We propose a model that is able to capture this difference among locations and times. 

\item We demonstrate effectiveness of the proposed model using the Foursquare dataset. The result shows the proposed method significantly outperforms the naive trajectory overlap based solution in prediction accuracy.

\end{itemize} 

The rest of the paper...

  
\section{Related Work} \label{sec:related}


\section{Overview} \label{sec:overview}

\subsection{Problem Statement} \label{sec:problem}


In this paper, we assume the precise check-ins, instead of geographic coordinates is given, as in the Foursquare dataset. Nevertheless, the proposed method is also applicable to coordinates...


We formally define the notion of \textit{Check-in} and \textit{User Trajectory} used in this paper.




We define a \textit{Co-visitation} of two users as follows: 



We formulate the social connection prediction problem as a classification problem, where the goal is to assign a given pair of users into one of the two classes: \textit{Friends} and \textit{Non-friends}.


\subsection{Methodology} \label{sec:framework}

What model to use?? DNN? Linear? 
Use feature selection to select only significant locations/time slots? AdaBoost?
Matrix Factorization? for full paper...

Given the trajectories of two users $u_i$ and $u_j$, we propose a three step method...

1. Feature extraction
2. Model learning
3. Prediction



Hypothesis (For a full paper), Research questions
1. Active user vs Non-active users?
2. Percentage of friends can be explained? 
3. Most significant types of locations?
4. Most significant time slots?
5. Geo-distance and predictive power?
%\section{Co-visitation Feature Extraction} \label{sec:features}
 

\section{Feature Selection and Model Learning} \label{sec:model}


\section{Experiments on Real-world Data} \label{sec:experiments}

\subsection{Dataset Description}

We evaluate the proposed techniques on the widely-used Foursquare check-in dataset~\cite{yang2015nationtelescope}. In our experiments, we mine the check-in data from two of the most popular cities, including New York City (NYC) and Tokyo. The dataset contains about 227,428 check-ins reported in NYC and 573,703 in Tokyo. The check-ins were collected for about 10 month. From each check-in, we extract the user ID, location ID, and a timestamp. Using the user ID or location ID, we retrieve the profile of that user or location on Foursquare. The user profile include the social connection between users (``follower - followee") and the location profile includes its category (\textit{Food, Coffee, Nightlife, Fun, and Shopping}), coordinates, and user rating. The check-ins are grouped by user ID/location ID and sorted by their timestamps.


\subsection{Experiment Design and Results}


\section{Conclusion} \label{sec:conclusion}

In this paper, we study the predictive power of user trajectories in reflecting their social connections. Based on the hypothesis that friends tend to visit same locations at same time, we propose to model the probability of that social connection exists between two users using their co-visitations. We propose a three step model learning process. First, a co-visitation feature vector is generated based on the trajectory of two users. We then employ feature selection to filter out less significant co-visitations from the feature vector. Finally, we explore several statistic models in order to find the one that is able to yield the best performance for our problem. In our preliminary experiments using a subset of users selected from the Foursquare dataset, we find the proposed methodology shows promising performance, in terms of prediction accuracy, comparing with a naive solution based on simple counting the number of co-visitations between users. As for future work, we aim to develop a more comprehensive social connection prediction framework that combines spatio-temporal data with network structure as well as node attribute.




\bibliographystyle{ACM-Reference-Format}
\bibliography{ref}

\balance
\end{document} 