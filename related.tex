\section{Related Work} \label{sec:related}

The spatio-temporal social connection prediction problem we study in this paper is directly related to link prediction problem on social networks. Given the snapshot of a social network at time $t$, the goal of link prediction is to predict links, i.e., social connections, that will emerge at a later time, or to identify missing links at $t$. Such missing links could be the result of privacy settings, e.g., a user may want to hide his friend list from the general public. 

Existing works in the field mainly explore two types of information in predicting links: 1) Network structure, i.e., existing social connections, and 2) node attributes such as user profiles. We briefly summarize some representative works. The relational learning~\cite{pieter2003link, yu2006stochastic, miller2009nonparametric} and matrix factorization-based~\cite{menon2011link} techniques both leverage attribute information for link prediction. The Supervised Random Walk (SRW) technique proposed in~\cite{backstrom2011supervised} combines networks structure and edge attributes to improve prediction accuracy, but does not fully explore node attributes. In~\cite{yin2010linkrec}, network structure and node attributes are integrated with a Social Attribute Network (SAN) model, which is later generalized in~\cite{gong2014joint} to both predict links and infer missing attributes.

Our problem is also closely related to~\cite{xiao2010finding}, which proposes to explore trajectory data to identify similar users. Their goal is to find users who share similar interests in locations, which serves as a friend recommendation tool on LBSN. The proposed technique employs clustering over users location history to identify similar users. In contrast, our work focus on studying the predictive power of trajectories in terms of reflecting existing or missing social connections among users. And our methodology is to model the probability of the existence of such social connections between two users. Form this perspective, our work intends to complement existing studies on human mobility patterns.