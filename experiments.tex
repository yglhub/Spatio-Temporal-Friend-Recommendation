\section{Evaluation} \label{sec:evaluation}

In this section, we report the preliminary experiment results of the proposed methodology on the trajectory of a selected subset of Foursquare users.

\subsection{Dataset Description}

We evaluate the proposed model on the widely-used Foursquare check-in dataset~\cite{yang2015nationtelescope}. In our experiments, we mine the check-in data from two of the most popular cities, including New York City (NYC) and Tokyo. The dataset contains about 227,428 check-ins reported in NYC and 573,703 in Tokyo. The check-ins were collected for about 10 month. From each check-in, we extract the user ID, location ID, and a timestamp. Using the user ID or location ID, we retrieve the profile of that user or location on Foursquare. The user profile include the social connection between users (``follower - followee") and the location profile includes its category (\textit{Food, Coffee, Nightlife, Fun, and Shopping}), coordinates, and user rating. The check-ins are grouped by user ID/location ID and sorted by their timestamps.


Select most active users...Ground truth...etc.

\subsection{Experiment Design}

For comparison purpose, we have implemented the following schemes:
\begin{itemize}
\item \textbf{Random} This scheme randomly assigns a user pair as friends or non-friends, each with a probability of 50\%.

\item \textbf{Naive} This scheme simple counts the number of co-visitations of two users. If the number is higher than a threshold, the two users are predicted to be friends, and otherwise non-friends. The threshold is set to be the average number of co-visitations of each pair of friends in the dataset we used.

\item \textbf{Proposed} The proposed model that assumes co-visitations occurred at different locations and time slots have different predictive power.
\end{itemize}

\subsection{Results}

