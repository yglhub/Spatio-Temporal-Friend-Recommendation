\section{Introduction}\label{sec:intro}

In the past decade, with the rise of Location-Based Social Networks (LBSN), huge amount of geo-spatial data is collected on a daily basis. For example, the Foursquare\cite{yang2015nationtelescope} dataset contains more than 30 millions of self-reported check-ins from thousands of user around the world. As a result, it becomes possible to mine spatio-temporal data and study human mobility pattern at unprecedented large scale. 

It is long known that a user's mobility pattern can be affected by his social connections~\cite{cho2011friendship, ye2013s}. For example, a group of close friends tend to check-in to the same locations at the same time period. As such, it is possible to predict a user's future movement by mining the historical trajectory of his friends on LBSN. In the past decade, making predictions with spatio-temporal data has been intensively studied. Existing research mainly focus on the prediction of future movements (e.g., \cite{cho2011friendship, noulas2012mining, gao2012mobile, scellato2011nextplace, lian2013collaborative}). To our knowledge, however, predicting a user's social connections with spatio-temporal data has not been studied in literate. 

Towards the goal of a more thorough understanding of human mobility patterns, we propose to investigate the predictive power of spatio-temporal data in predicting a user's social connections. In particular, given the trajectories of two LBSN users $u_i$ and $u_j$, we aim to predict the probability that $u_i$ and $u_j$ are friends on the LBSN. Social connection prediction is a long standing research topic, mainly used as a tool for friend recommendation on social networks. Most existing methods exploit a user's profile and existing social connections to make friend recommendations, but not the user's trajectory. Our research is not competitive, but supplementary to existing friend recommendation methods.

A straightforward way to predict the social connection, or the lack thereof, between two users is to exam the \textit{spatio-temporal overlap} of their trajectories, i.e., find events where the two users visit the same location at the same time on their trajectories. Such an event is called a \textit{co-visitation} of the two users. The assumption is, if two users frequently visits the same location during the same time peroid, they might be friend with each other. Thus the number of co-visitations could reflect when and where they were meeting. Algorithms such as co-location mining~\cite{weiler2015geo} can be used to discover such co-visitations among users.

Although the above assumption is reasonable, this naive solution suffers from two problems. First, it treats all locations equally in predicting social connections, which is not realistic. For example, if two users frequently meet at private locations like someone's house, or a small coffee shop, it is very likely that they know each other. However, if they both check-in to the same Walmart supermarket after work, it might be just an coincidence simply because there it is the only supermarket near their home. Second, this method ignores the time difference of check-ins behaviours. If two users both check-in to a restaurant at 6:00pm, it is not as significant as two users visit the same location at 10:00pm. This is because most customer of the restaurant may choose to dine there around 6:00pm, but if two users both decide to check-in there at 10:00pm, the chance that it is purely an coincidence is relatively lower.

We propose to employ a more comprehensive methodology to study the social connection prediction problem. Unlike the naive solution, we assume different locations and different time slots have different predictive power. We propose a social connection prediction model that is able to capture the inherent difference among locations and times. Specifically, each location and time slot is assigned a weight which measures the significance of the location and time slot in predicting social connections. We then use the Foursquare dataset to learn the weights in the proposed model and make predictions using users' social connections on Foursquare as ground truth. We show that the proposed model significantly outperforms the naive algorithm that exams only the number of co-visitations. We summarize our contributions as follows:
\begin{itemize}

\item We study the predictive power of spatio-temporal data in predicting social connections. Given the trajectories of two LBSN users $u$ and $v$, we aim to predict the probability that $u$ and $v$ are friends on the LBSN.

\item We assume different locations and time may have different predictive power, which is in accordance with common sense. We propose a model that is able to capture this difference among locations and times. 

\item We demonstrate effectiveness of the proposed model using the Foursquare dataset. The result shows the proposed method significantly outperforms the naive trajectory overlap based solution in prediction accuracy.

\end{itemize} 

The rest of the paper...

  