\section{Introduction}\label{sec:intro}

In the past decade, with the rise of Location-Based Social Networks (LBSN), huge amount of geo-spatial data is collected on a daily basis. For example, the Foursquare\cite{yang2015nationtelescope} dataset contains more than 30 millions of self-reported check-ins from thousands of user around the world. As a result, it becomes possible to mine spatio-temporal data and study human mobility pattern at unprecedented large scale.

Several studies on human mobility pattern reveal that a user's movement can be affected by his social connections~\cite{cho2011friendship, ye2013s}. For example, a group of close friends tend to check-in to the same locations at the same time period. As such, it is possible to predict a user's future movement by mining the historical trajectory of his friends on LBSN. These studies has since inspired a seriers of research efforts towards the prediction of future individual movements (e.g., \cite{noulas2012mining, gao2012mobile, scellato2011nextplace, lian2013collaborative}). Another research direction (e.g., \cite{xiao2010finding}) focuses on exploring historical trajectories to identify similar users, which can be used for friend recommendation and community detection on LBSN. 


Towards the goal of a more thorough understanding of human mobility patterns, we propose to investigate the inverse problem:  How does a set of users' trajectory reflects their social connections. In particular, we define the social connection prediction problem: Given the trajectories of two LBSN users $u_i$ and $u_j$, we aim to model the probability that $u_i$ and $u_j$ are friends on the LBSN using their trajectories. Social connection prediction is a long standing research topic. Most existing methods relies on link prediction techniques, which exploit a user's profile and existing social connections to make predictions of hiding links between users, but not users' trajectory. The focus of this paper is not to compete with, but to supplement existing methods.

A straightforward way to predict the social connection, or the lack thereof, between two users is to exam the \textit{spatio-temporal overlap} of their trajectories, i.e., find events where the two users visit the same location at the same time on their trajectories. We define such an event as a \textit{co-visitation} of the two users. The assumption is, if two users frequently visits the same location during the same time peroid, they might be friend with each other. Thus the occurance of co-visitations could reflect when and where they were meeting. The same assumption is used to identify similar users in~\cite{xiao2010finding}. Algorithms such as co-location mining~\cite{weiler2015geo} can be used to discover co-visitations among users. 

Although the above assumption is reasonable, this naive solution suffers from two problems. First, it treats all locations equally in predicting social connections, which is not realistic. For example, if two users frequently meet at private locations like someone's house, or a small coffee shop, it is very likely that they know each other. However, if they both check-in to the same Walmart supermarket after work, it might be just an coincidence simply because there it is the only supermarket near their home. Second, this method ignores the time difference of check-ins behaviours. If two users both check-in to a restaurant at 6:00pm, it is not as significant as two users visit the same location at 10:00pm. This is because most customer of the restaurant may choose to dine there around 6:00pm, but if two users both decide to check-in there at 10:00pm, the chance that it is purely an coincidence is relatively lower. Although the technique proposed in~\cite{xiao2010finding} considered the impact of different granularity of locations (e.g., the same state v.s. the same city), it does not explicitly distinguish the predictive power of different locations.

We propose to employ a more comprehensive methodology to study the social connection prediction problem. Unlike the naive solution, we assume different locations and different time slots have different predictive power. We propose a social connection prediction model that is able to capture the inherent difference among locations and times. Specifically, we first convert the trajectory of two users into a high-dimensional co-visitation feature vector, and then use feature selection to filter out less significant co-visitations. A predictive model is then learnt based on the selected features using the Foursquare dataset. Using the users' social connections on Foursquare as ground truth, we show that the proposed model outperforms the naive algorithm that counts only the number of co-visitations. We summarize our contributions as follows:
\begin{itemize}

\item We study how the trajectories of a set of users reflect their social connections. To this end, we define the social connection prediction problem: Given the trajectories of two LBSN users $u$ and $v$, we aim to model the probability that $u$ and $v$ are friends on the LBSN.

\item Our key observation is: different locations and time may have different predictive power, which is in accordance with common sense. As such, we propose a social connection model that is able to capture this difference among locations and times. 

\item We demonstrate effectiveness of the proposed model using the Foursquare dataset. The result shows the proposed method outperforms the naive trajectory overlap based solution in prediction accuracy.

\end{itemize} 

The rest of the paper is organized as follows: Related works are summarized in Section 2. We formally define our problem and give an overview of our methodology in Section 3. Section 4 discuses the feature extraction process. Section 5 presents details of model training. Experiment results are showed and analysed in Section 6. And finally, Section 7 concludes the paper.
