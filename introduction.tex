\section{Introduction}\label{sec:intro}

In the past decade, with the rise of Location-Based Social Networks (LBSN), huge amount of geo-spatial data is generated on a daily basis. As a result, it becomes possible to mine spatio-temporal data at unprecedented large scale. For example, the Foursquare\cite{yang2015nationtelescope} dataset contains more than 30 millions of self-reported check-ins from thousands of user around the world. 

It is long known that a user's mobility pattern can be affected by his social connections~\cite{cho2011friendship, ye2013s}. For example, a group of close friends tend to check-in to the same locations at the same time period. As such, it is possible to predict a user's future movement by mining the historical trajectory of his friends on LBSN. Towards the goal of a more thorough understanding of human mobility patterns, we propose to investigate the inverses problem: How does a user's trajectory reflects his social connections. To answer this question, we study the predictive power of spatio-temporal data in predicting a user's social connections. In particular, given the trajectories of two LBSN users $u$ and $v$, we aim to predict the probability that $u$ and $v$ are friends on the LBSN.

Making predictions with spatio-temporal data has been intensively studied, but mostly for the prediction of future movements (e.g., \cite{cho2011friendship, noulas2012mining, gao2012mobile, scellato2011nextplace, lian2013collaborative}). To our knowledge, however, making predictions on the user's social connections has not been studied in literate. We aim to complement existing research in this field by addressing this issue. Predicting social connections with spatio-temporal data also provides a supplement method for \textit{friend recommendation} on LBSN. It can be used to discover users that potentially share the same interests for locations such as restaurants, shops, and museums.

A straightforward way solution to the social connection prediction problem is to find the spatio-temporal overlap of two trajectories. The assumption is, if two users frequently visits the same location during the same time peroid on their trajectories, they might be friend with each other. And the trajectory overlapping reflects the time and locations when and where they were meeting. Algorithms such as co-location mining~\cite{weiler2015geo} can be used to detect such spatio-temporal overlap among users. 

Although the assumption is reasonable, this naive solution suffers from two problems. First, it treats all locations equally in predicting social connections, which is not realistic. For example, if two users frequently meet at private locations like someone's house, or a small coffee shop, it is very likely that they know each other. However, if they both check-in to the same Walmart supermarket after work, it might be just an coincidence simply because there it is the only supermarket near their home. Second, this method ignores the time difference of check-ins behaviours. If two users both check-in to a restaurant at 6:00pm, it is not as significant as two users visit the same location at 10:00pm. This is because most customer of the restaurant may choose to dine there around 6:00pm, but if two users both decide to check-in there at 10:00pm, the chance that it is an coincidence is much lower.

We propose a spatio-temporal social connection prediction framework that aims at address the above problems. Unlike the naive solution, we assume different locations and different time slots have different predictive power, which is modelled as latent variables. We first formulate the social connection prediction problem as a latent variable estimation problem, and then design a matrix factorization-based algorithm to estimate the variables and make predictions. We show that the proposed framework, which takes into consideration of both spatial and temporal difference in making predictions, significantly outperforms the naive algorithm. We summarize our contributions as follows:
\begin{itemize}

\item We introduce the spatio-temporal social connection prediction problem. Given the trajectories of two LBSN users $u$ and $v$, we aim to predict the probability that $u$ and $v$ are friends on the LBSN.

\item We formulate the social connection prediction problem as matrix factorization problem. The predictive power of different locations and time are treated as latent variables and estimated using Matrix Factorization.

\item We demonstrate effectiveness of the proposed framework using the Foursquare dataset. The result shows the proposed method significantly outperforms the naive trajectory overlap based solution in prediction accuracy.

\end{itemize} 

The rest of the paper...

  