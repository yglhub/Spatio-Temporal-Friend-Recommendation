\section{Overview} \label{sec:overview}

\subsection{Problem Statement} \label{sec:problem}

We define a user's trajectory as a series of timestamped check-ins, where each check-in indicates the exact place (i.e., a restaurant, a coffee shop, etc.) the user visits, instead of a geo-graphical coordinate. The Foursquare dataset is an example of such trajectories that consists of self-reported check-ins. Note that coordinate-based trajectory can be converted into such check-ins by joining the coordinates with a database of Point-of-Interests (PoI), such as provided by Open-Stree Map. Therefore, for simplicity, we consider only check-in-based trajectory in this paper. We formally define the notion of \textit{Check-in} and \textit{User Trajectory}.

\begin{definition}[Check-in]
Let $U$ denote a set of unique user identifiers, $L$ denote a set of locations, and $T$ denote the time domain. A check-in $c$ is a triple $(u, l, t) \in U \times L \times T$, which indicates the user $u$ has visited $l$ at time $t$.
\end{definition}

\begin{definition}[User-trajectory]
Let $C$ be a collection of check-ins and $u \in U$ a user, then the set $C_u := \{ (u', l, t) \in C | u = u'\}$ is the user-trajectory (or simply trajectory) of $u$.
\end{definition}

The proposed social connection prediction model is based on the concept of \textit{Co-visitation}, which is defined as follows: 

\begin{definition}[Co-visitation]
Two users $u_i$ and $u_j$ co-visit a location $l$ if $u_i$ and $u_j$ report two check-ins $(u_i, l, t_i)$ and $(u_j, l, t_j)$ respectively, where $| t_i - t_j| \leq \tau$.
\end{definition}

Here $\tau$ is an experience-based parameter call the minimal co-visitation time window. We formulate the social connection prediction problem as a classification problem. Given the trajectory and profile of two users $u_i$ and $u_j$, the goal is to assign the pair of users $(u_i, u_j)$ into one of the two classes: \textit{Friends} or \textit{Non-friends}. 

\subsection{Methodology} \label{sec:framework}

We propose to model the probability of the existence of social connection between two users based on the hypothesis that socially connected users tend to visit same locations at same time periods, and such events are defined as co-visitations.  

What model to use?? DNN? Linear? 
Use feature selection to select only significant locations/time slots? AdaBoost?
Matrix Factorization? for full paper...

Given the trajectories of two users $u_i$ and $u_j$, we propose a three step method...

1. Feature extraction
2. Model learning
3. Prediction



Hypothesis (For a full paper), Research questions
1. Active user vs Non-active users?
2. Percentage of friends can be explained? 
3. Most significant types of locations?
4. Most significant time slots?
5. Geo-distance and predictive power?