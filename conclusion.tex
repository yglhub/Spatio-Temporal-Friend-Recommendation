\section{Conclusion} \label{sec:conclusion}

In this paper, we study the predictive power of user trajectories in estimating their social connections. Based on the hypothesis that friends tend to visit same locations at same time, we propose to model the probability of that social connection exists between two users using their co-visitations. We propose a three step model learning process. First, a co-visitation feature vector is generated based on the trajectory of two users. We then employ feature selection to filter out less significant co-visitations from the feature vector. Finally, we explore several statistic models in order to find the one that is able to yield the best performance for our problem. In our preliminary experiments using a subset of users selected from the Foursquare dataset, we find the proposed methodology shows promising performance, in terms of prediction accuracy, comparing with a naive solution based on simple counting the number of co-visitations between users. As for future work, we aim to develop a more comprehensive social connection prediction framework that combines spatio-temporal data with network structure as well as node attribute.

